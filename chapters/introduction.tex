\chapter*{Introduction}
\addcontentsline{toc}{chapter}{Introduction}

When Apple made their Swift language open source in 2015 giving it cross-platform support, Swift developers started putting their efforts into shifting some of Swift's focus to the server side. 
Given Swift's strict and powerful type-system, efficient and safe memory management and modern concurrency model, it is not only a great tool for making client side apps, but also a perfect fit for the server. That's why Tanner Nelson created Vapor, currently the most popular server side swift framework. As with every respectable server-side framework, Vapor too needed a safe and efficient JWT library. JSON Web Tokens, or JWTs, are the go-to technology for enabling secure communication between parties. This is where JWTKit comes in. Since its creation, JWTKit has been a crucial library for Swift on the server developers and has been long due for a re-evaluation. 

JWTKit was initially constructed as a wrapper for Google's BoringSSL, a famous C library based on top of OpenSSL. This approach worked well for the first few versions up until version 4. Maintaining a Swift library dependent on C is not modern, safe nor efficient, this is why the decision was made to eradicate BoringSSL from JWTKit. With Swift being run on Linux more and more, Apple published an open source version of CryptoKit, SwiftCrypto, which provides cryptographic operations to Apple devices. Being maintained by Apple, available on Linux and having a C free API makes SwiftCrypto the best companion for JWTKit.

Additionally, with Swift's new concurrency system, maintaining a library which was not thread-safe by default didn't make sense, that's why during JWTKit's v5 redesign another one of the upgrades the package got was integration with the Sendable protocol and Swift 5.5's new concurrency model, which guarantees thread-safety and eliminates data races from the library's types. Then, the missing RSA-PSS algorithm was added for signing and verifying tokens, in addition to the already present RSASSA-PKCS1-v1\_5. X5C header verification, which was previously based on BoringSSL too, was switched over to use Apple's swift-certificates package, which allows working with X.509 Certificates. Finally, the possibility was added for users to implement their own parsing and serialising of the tokens, allowing for custom header fields like the zip header, which modifies the structure of the token by compressing the payload.

All of these changes and more are explained in detail in this thesis after an introduction to Swift and the server side Swift ecosystem, the JSON Web Token technology, and how these two merge. Finally the thesis will explore how the updates impact performance relatively to the previous version.