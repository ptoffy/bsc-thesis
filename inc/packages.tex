% !TeX root = ../main.tex
\usepackage{a4wide}

\usepackage[utf8]{inputenc}

\usepackage[english]{babel}

\usepackage{graphicx}  % To include graphics
\usepackage{titlesec}  % To customize titles
\usepackage{titling}   % To customize the title page
\usepackage{titlepic}  % To insert a picture in the title page
\usepackage{fontspec}  % For pretty fonts
\usepackage{minted}    % Code highlighting
\usepackage{amsfonts}  % To include maths symbols
\usepackage{enumitem}  % Style itemize items
\usepackage{multicol}  % To have multiple columns in a page

\usepackage{mdframed}
\usepackage{listings} % To include code
\usepackage{geometry} % To set the dimensions of the document
\usepackage{url} % To use URLs in the bibliography
\usepackage{hyperref} % To reference sections in text
\usepackage[acronym]{glossaries}
\usepackage{glossary-mcols}

% Code settings 

\usepackage{lstautogobble}                  % Fix relative indenting
\usepackage[usenames,dvipsnames,svgnames]{xcolor}    % Code coloring
\usepackage{zi4}                            % Nice font
\usepackage{booktabs}

\usepackage{pgfplots}
\pgfplotsset{compat=1.17} 

\definecolor{bluekeywords}{rgb}{0.13, 0.13, 1}
\definecolor{greycomments}{gray}{0.5}
\definecolor{redstrings}{rgb}{0.9, 0, 0}
\definecolor{graynumbers}{rgb}{0.5, 0.5, 0.5}
\definecolor{darkergreen}{rgb}{0,0.5,0}
\definecolor{darkerred}{rgb}{0.9,0,0}
% \definecolor{purplekeywords}{rgb}{0.6274509803921569, 0.27058823529411763, 0.6235294117647059}

\usemintedstyle{manni}
\newmintinline{latex}{}
\newmintinline{swift}{}

\setmonofont{Inconsolata}

\setminted {
    autogobble,
    breaklines=true,
    breakanywhere=false,
    frame=leftline,
    framesep=12pt,
    numbersep=5pt,
    tabsize=4,
    escapeinside=||
}

\newminted{swift}{
    autogobble,
    breaklines=true,
    breakanywhere=false,
    frame=leftline,
    framesep=12pt,
    numbersep=5pt,
    tabsize=4,
    escapeinside=||
}

\lstset{
    language=Swift,
    autogobble,
    columns=fullflexible,
    showspaces=false,
    showtabs=false,
    breaklines=true,
    showstringspaces=false,
    breakatwhitespace=true,
    escapeinside={(*@}{@*)},
    keywordstyle=\color{RubineRed}\bfseries,
    morecomment=[l]{//},
    commentstyle=\color{greycomments},
    stringstyle=\color{redstrings},
    numberstyle=\color{blue},
    basicstyle=\ttfamily,
    frame=l,
    framesep=12pt,
    xleftmargin=12pt,
    tabsize=4,
    captionpos=b,
}

% used to create markdown-like quotes
\newmdenv[
  % leftmargin=10pt,
  % rightmargin=10pt,
  innerleftmargin=10pt,
  innerrightmargin=10pt,
  linecolor=gray,
  backgroundcolor=Gainsboro!50,
  topline=false,
  bottomline=false,
  rightline=false,
  linewidth=2pt
]{customblockquote}

\newcommand{\note}[2]{
  \begin{customblockquote}
    \textbf{#1} \vspace{0.2em} \\ 
    #2
  \end{customblockquote}
}


% Command to create a glossary entry with correspondent acronym.
% Args: 1: identifier, 2: acronym/name, 3: long name, 4: description
\newcommand{\newglossaryentrywithacronym}[4]{
    %%% The glossary entry the acronym links to   
    \newglossaryentry{#1_gls}{
        name={#3}, % Acronym as it appears in the glossary
        description={#4} % Description
    }

    % Acronym pointing to glossary
    \newglossaryentry{#1}{
        type=\acronymtype,
        name={#2}, % Acronym name
        description={#3}, % Long name
        first={#3 (#2)\glsadd{#1_gls}}, % How it first appears: long name (acronym)
        see=[Glossary:]{#1_gls} % Link to the full glossary entry
    }
}